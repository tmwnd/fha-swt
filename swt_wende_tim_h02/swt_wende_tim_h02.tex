\documentclass{article}
\usepackage[a4paper, margin=5em]{geometry}
\usepackage{ngerman}
\usepackage{fancyhdr}
\usepackage{lastpage}
\usepackage{enumitem}
\usepackage{hyperref}

\setlength{\parindent}{0em}
\newcommand{\gqq}[1]{\glqq{}#1\grqq{}}

\pagestyle{fancy}
\fancyhf{}
\renewcommand{\headrulewidth}{0pt}
\fancyfoot{}

\lfoot{}
\cfoot{Seite \thepage\ / \pageref*{LastPage}}
\rfoot{}

\hypersetup{
    colorlinks=true,
    linktoc=all,
    urlcolor=blue
}

\author{Tim Wende}
\date{\today}
\title{\textbf{Hausaufgabe 4}}

\begin{document}
    \maketitle

    \newpage
    \section{Boehm’sches Spiralmodell}
    In der Vorlesung wurde ein wichtiger Meilenstein bei der Entwicklung von Software-Entwicklungsprozessen ausgelassen, das so genannte Boehmsche Spiralmodell.
    Informieren Sie sich über dieses Modell und beschreiben Sie die Hauptideen des Ansatzes.
    Überlegen Sie, in wie weit es sich um ein iteratives Modell und dann um ein inkrementelles Modell handelt.
    Dokumentieren Sie ihre Nachforschungen und Überlegungen.
    
    \vspace{1em}
    Die Hauptidee des Boehm’schen Spiralmodells ist es das größte Risiko zu beseitigen.
    So wird jeder Zyklus des Wasserfall-Modells iterativ mehrfach durchlaufen.
    Dadurch nähert dich das Projekt über einen längeren Zeitraum den jeweiligen Zielen an.
    Hierbei finden vermehrt Risikoanalysen wie Verifikationen und Validationen statt.

    \vspace{1em}
    Da die Phasen iterativ durchlaufen werden, kann man das Boehm’sche Spiralmodell als iterative Vorgehensweise beschreiben.
    Jedoch ist nach einem Zyklus die Phase abgeschlossen, also ein Inkrement, beispielsweise ein Prototyp, fertig.
    Dadurch vermischt es diese beiden Vorangehensweisen miteinander.

    \vspace{1em}
    Dokumentation meiner Nachforschung:\\
    \href{https://de.wikipedia.org/wiki/Spiralmodell#/}{Spiralmodell} [15.10.2021]

    \newpage
    \section{Veränderte Anforderungen}
    Einem Kunden fällt nach Zweidritteln erfolgreicher Projektlaufzeit ein, dass er eine bereits entwickelte Funktionalität nicht benötigt, dafür aber eine andere wünscht.
    Beschreiben Sie kurz, wie man bei Nutzung der folgenden Vorgehensmodelle darauf reagieren würde:

    \begin{enumerate}[label=\alph*.]
        \item Wasserfallmodell
        \label{item:Wasserfallmodell}
        
            weinen?\\
            Das Projekt als gescheitert ansehen / stornieren?\\
            Den Kunden erschießen?

            Bleibt dir ja nicht viel übrig, nicht wahr?

            In der Realität würde man im \gqq{quick and dirty} style Funktionen rauslöschen, merken, dass man sie aus komplett nicht nachvollziehbaren Gründen noch benötigt, code Leichen in Kauf nehmen und die vorhandene Basis komplett zweckentfremden.
            Zusätzlich würde man komplett planlos, gar wahrlos, Funktionen hinzufügen und dem Kunden ein unsauberes unfertiges Projekt an den Kopf schmeißen.
            Nach diesem \gqq{Erfolg} sollte der Kunde das Unternehmen am besten meiden (spätesten, wenn er von der vorhanden-sein des Wasserfall-Modells hört).

        \item Iterative Entwicklung
        
            Man kann hoffen, dass das Projekt noch nicht weit vorangeschritten ist, und man die fehlerhafte Abzweigung rechtzeitig entdeckt und \gqq{geradebiegen} kann.
            Oh wait, \gqq{Einem Kunden fällt nach Zweidritteln erfolgreicher Projektlaufzeit ein,  [\ldots]}\\
            Na dann: siehe \ref{item:Wasserfallmodell})
        
        \item Inkrementelle Entwicklung
        \label{item:Inkrementelle_Entwicklung}
        
            Man kann hoffen, dass die Funktion in einem eigenen Inkrement entwickelt wurde, welches komplett unabhängig von den anderen funktioniert.
            Ist unwahrscheinlich, aber hey, wenigstens ein bisschen Hoffnung bleibt.
            Dieses Inkrement kann gelöscht werden und weitere Inkremente werden \gqq{hinten} an die Ablaufstruktur angefügt.
            So kann man ein halbwegs sauberes Projekt erfolgreich beenden und den Kunden von fortan meiden.
            Sollte dies nicht so glücklich voranschreiten: siehe \ref{item:Wasserfallmodell})

        \item Wo würden Sie das SCRUM Modell einordnen?
        
            Im Zweifel lassen sich mehrere Sprint-Teams dafür missbrauchen die aktuell \gqq{fehlerhaften} Funktionalitäten zu beseitigen, sowie die neu gewünschten zu implementieren.
            Falls nicht: siehe \ref{item:Inkrementelle_Entwicklung}), da die äußere Vorgehensweise in SCRUM hauptsächlich iterativ arbeitet.
        
            Auf jeden Fall werden in einigen solcher Fälle viele Teamleiter schwitzen sowie viele viele Überstunden entstehen.
    \end{enumerate}

    \newpage
    \section{SCRUM}
    Suchen Sie im Netz nach dem offiziellen Scrum-Guide von Ken Schwaber und Jeff Sutherland.
    Lesen Sie die aktuelle Version (englische Variante) und beantworten Sie anschließend die folgenden Fragen. (Hier zu finden: \href{https://scrumguides.org/}{Scrum Guides})

    \begin{enumerate}[label=\alph*.]
        \item Welche Personen/Rollen sieht das Scrum-Modell vor?
        
            \emph{Developer, Product Owner, Scrum Master}
        \item Welcher Zeitraum ist laut Dokumentation vorgesehen für die Planung eines Sprints?
        
            $\leq$ 8 hours
        \item Welcher Zeitraum ist laut Dokumentation vorgesehen für die folgenden Abläufe / Meetings:
        
            \begin{itemize}[label=◦]
                \item Daily Scrum:
                
                $\approx$ 15 minutes
                \item Spring:
                
                $\leq$ 1 month
                \item Spring Retroperspektive:
                
                $\leq$ 3 hours
                \item Spring Review:
                
                $\leq$ 4 hours
            \end{itemize}
        \item Welche Nachteile und welche möglichen Probleme bringt das Vorgehen nach Scrum mit sich?
        
            Zu viel Vertrauen in- und Verantwortung für- die Mitarbeitenden.
            Da man nur in einem Sprint gleichzeitig sein sollte, wird Fachpersonal für eine spezifische Aufgabe \gqq{verschwendet} und Team übergreifendes arbeiten erschwert.
            Planung ist gut; zu viel Planung kann einem im Weg stehen $\to$ SCRUM ist offensichtlich zweiteres.
    \end{enumerate}
\end{document}